\documentclass[12pt,a4paper]{article}
\usepackage{a4}
\usepackage[utf8]{inputenc}
\usepackage[ngerman]{babel}
\usepackage{graphicx}				%Grafiken einbinden
\usepackage{amsfonts,amstext,amsmath}		%Benötigt für align-Umgebung
\usepackage{pdfpages}				
\usepackage[bf,font=footnotesize]{caption} 	%Bildunterschrift
\usepackage{multicol}				%Text in mehrere Spalten unterteilen
\usepackage{url}
\usepackage{diagbox}
\usepackage{multirow}
\usepackage{subfigure}
\renewcommand{\*}{\cdot}			
\renewcommand{\d}{\text{d}}
\begin{document}
\section{Theorie}
\subsection{Elektronenstrahl}
Ein Elektronenstrahl wird i.A., so auch bei diesem Experiment, mithilfe einer Glühkathode in Kombination mit einem Wehneltzylinder
erzeugt.
Zuerst eine sogenannte Heizspule mit der Heizspannung $U_H$ zum Glühen gebracht. Die dadurch freigesetzten Elektronen werden in einer
Zylinderkathode (Wehneltzylinder) auf den Mittelpunkt dieser fokussiert und dann durch eine Anodenplatte mit einem kleinen Loch in 
der Mitte beschleunigt.
\begin{center}
\begin{minipage}{5cm}
\includegraphics{Elektronenkanone.png}
\end{minipage}
\end{center}

\subsection{Helmholtzspule}
Die Helmholtzspule ist eine Apparatur zur Erzeugung eines homogenen Magnetfeldes. Das wird erreicht, indem zwei Leiterschleifen vom 
Radius $R$ parallel zueinander von einem Strom durchflossen werden. Für das Magnetfeld ergibt sich durch Symmetrie nur eine Abhängigkeit
von der $\hat{e}_z$-Achse, sodass sich das Magnetfeld ergibt zu:
\begin{align}
B_{z} & = \frac{1}{2} \mu_{0} \mu_{r} n I R^2 \left[ \left( R^2+ \left( z - \frac{R}{2} \right)^2 \right)^{- \frac{3}{2}} + \left( R^2 + \left( z + \frac{R}{2} \right)^2 \right)^{- \frac{3}{2}} \right] \text{.} 
\end{align}

\subsection{Herleitung $\frac{e}{m_e}$}
Zur Herleitung des Quotienten $\frac{e}{m_e}$ kann man das Kräftegleichgewicht betrachten. Dazu werden folgende zwei Kräfte betrachtet:
\begin{itemize}
\item Lorentzkraft $\vec{F}_L$
\item Zentripetalkraft $\vec{F}_Z$
\end{itemize}
Die Lorentzkraft ist die Kraft, die auf ein Elektron in einem $\vec{E}$- oder $\vec{B}$-Feld wirkt:
\begin{align*}
\vec{F}_L&=&q\left(\vec{E}+\vec{v}x\vec{B}\right)\text{.}
\end{align*}
Da wir bei unserem Experiment nur ein Magnetfeld betrachten, also $\vec{E}=0$, lässt sich die Lorentzkraft auch schreiben als:
\begin{align}
\vec{F}_L&=&q\vec{v}x\vec{B}.
\end{align}
Die Zentripetalkraft $F_Z$ hingegen ist die Kraft, die auf ein um einen festen Punkt rotierenden Körper wirkt:
\begin{align}
\vec{F}_Z&=&\frac{m_ev^2}{r}\text{.}
\end{align}
In unserem Fall sind beide Kräfte gleich, wenn das Elektron im $\vec{B}$-Feld im Kreis rotiert. In diesem Fall lässt sich der für
unser Experiment gesuchter Radius $r$ bestimmen.

\begin{align}
F_Z&=&F_L \\
\frac{m_ev^2}{r}&=&q\vec{v}x\vec{B}
\Rightarrow \frac{e}{m_e}&=&\frac{v}{Br}
\end{align}
Durch Einsetzen des Magnetfeldes ergibt sich für den Quotienten $\frac{e}{m_e}$:
\begin{align}
\frac{e}{m_e}&=&\frac{125U_BR^2}{32\left(r\mu_0\mu_rnI\right)^2}\text{.}
\end{align}
\end{document}